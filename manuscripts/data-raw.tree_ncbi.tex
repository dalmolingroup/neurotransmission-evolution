\texttt{}

\hypertarget{resources}{%
\paragraph{\texorpdfstring{\textbf{Resources}}{Resources}}\label{resources}}

\texttt{}\\

\begin{table}[H]

\caption{\label{tab:string_species}Lists all organisms in STRING v11.}
\begin{tabular}[t]{rllll>{\raggedright\arraybackslash}p{18em}}
\toprule
\multicolumn{6}{c}{\bgroup\fontsize{12}{14}\selectfont \cellcolor[HTML]{EEEEEE}{\ttfamily{\textbf{string\_species}}}\egroup{}} \\
\cmidrule(l{3pt}r{3pt}){1-6}
\# & Col. name & Col. type & Used? & Example & Description\\
\midrule
\rowcolor{gray!6}  1 & taxid & character & yes & 9606 & NCBI Taxonomy identifier\\
2 & string\_type & character & no & core & if the genome of this species is core or periphery\\
\rowcolor{gray!6}  3 & string\_name & character & yes & Homo sapiens & STRING species name\\
4 & ncbi\_official\_name & character & no & Homo sapiens & NCBI Taxonomy species name\\
\bottomrule
\multicolumn{6}{l}{\textbf{Location: } data-raw/download/species.v11.0.txt}\\
\multicolumn{6}{l}{\textbf{Source: } stringdb-static.org/download/species.v11.0.txt}\\
\end{tabular}
\end{table}
\begin{table}[H]

\caption{\label{tab:ncbi_merged_ids}Links outdated taxon IDs to corresponding new ones.}
\begin{tabular}[t]{rllll>{\raggedright\arraybackslash}p{18em}}
\toprule
\multicolumn{6}{c}{\bgroup\fontsize{12}{14}\selectfont \cellcolor[HTML]{EEEEEE}{\ttfamily{\textbf{ncbi\_merged\_ids}}}\egroup{}} \\
\cmidrule(l{3pt}r{3pt}){1-6}
\# & Col. name & Col. type & Used? & Example & Description\\
\midrule
\rowcolor{gray!6}  1 & taxid & character & yes & 140100 & id of node that has been merged\\
2 & new\_taxid & character & yes & 666 & id of node that is the result of merging\\
\bottomrule
\multicolumn{6}{l}{\textbf{Location: } data-raw/download/taxdump/merged.dmp}\\
\multicolumn{6}{l}{\textbf{Source: } ftp.ncbi.nlm.nih.gov/pub/taxonomy/taxdump.tar.gz}\\
\end{tabular}
\end{table}
\begin{table}[H]

\caption{\label{tab:ncbi_edgelist}Represents taxonomy nodes.}
\begin{tabular}[t]{rllll>{\raggedright\arraybackslash}p{18em}}
\toprule
\multicolumn{6}{c}{\bgroup\fontsize{12}{14}\selectfont \cellcolor[HTML]{EEEEEE}{\ttfamily{\textbf{ncbi\_edgelist}}}\egroup{}} \\
\cmidrule(l{3pt}r{3pt}){1-6}
\# & Col. name & Col. type & Used? & Example & Description\\
\midrule
\rowcolor{gray!6}  1 & taxid & character & yes & 2 & node id in NCBI taxonomy database\\
2 & parent\_taxid & character & yes & 131567 & parent node id in NCBI taxonomy database\\
\rowcolor{gray!6}  3 & rank & character & no & superkingdom & rank of this node\\
4 & ... & ... & no & ... & (too many unrelated fields)\\
\bottomrule
\multicolumn{6}{l}{\textbf{Location: } data-raw/download/taxdump/nodes.dmp}\\
\multicolumn{6}{l}{\textbf{Source: } ftp.ncbi.nlm.nih.gov/pub/taxonomy/taxdump.tar.gz}\\
\end{tabular}
\end{table}
\begin{table}[H]

\caption{\label{tab:ncbi_taxon_names}Links taxon IDs to actual species names.}
\begin{tabular}[t]{rllll>{\raggedright\arraybackslash}p{18em}}
\toprule
\multicolumn{6}{c}{\bgroup\fontsize{12}{14}\selectfont \cellcolor[HTML]{EEEEEE}{\ttfamily{\textbf{ncbi\_taxon\_names}}}\egroup{}} \\
\cmidrule(l{3pt}r{3pt}){1-6}
\# & Col. name & Col. type & Used? & Example & Description\\
\midrule
\rowcolor{gray!6}  1 & taxid & character & yes & 2 & the id of node associated with this name\\
2 & name & character & yes & Monera & name itself\\
\rowcolor{gray!6}  3 & unique\_name & character & no & Monera <bacteria> & the unique variant of this name if name not unique\\
4 & name\_class & character & yes & scientific name & type of name\\
\bottomrule
\multicolumn{6}{l}{\textbf{Location: } data-raw/download/taxdump/names.dmp}\\
\multicolumn{6}{l}{\textbf{Source: } ftp.ncbi.nlm.nih.gov/pub/taxonomy/taxdump.tar.gz}\\
\end{tabular}
\end{table}

\texttt{}

\hypertarget{updating-string-taxon-ids}{%
\paragraph{\texorpdfstring{\textbf{Updating STRING taxon
IDs}}{Updating STRING taxon IDs}}\label{updating-string-taxon-ids}}

\texttt{}\\
Some organisms taxon IDs are outdated in STRING. We must update them to
work with the most recent NCBI Taxonomy data.

\begin{Shaded}
\begin{Highlighting}[]
\NormalTok{string_species }\OperatorTok
\StringTok{  }\KeywordTok{left_join}\NormalTok{(ncbi_merged_ids) }\OperatorTok
\StringTok{  }\KeywordTok{mutate}\NormalTok{(}\DataTypeTok{new_taxid =} \KeywordTok{coalesce}\NormalTok{(new_taxid, taxid))}
\end{Highlighting}
\end{Shaded}

\hypertarget{creating-tree-graph}{%
\paragraph{\texorpdfstring{\textbf{Creating tree
graph}}{Creating tree graph}}\label{creating-tree-graph}}

\texttt{}\\
The first step is to create a directed graph representing the NCBI
Taxonomy tree.

\begin{Shaded}
\begin{Highlighting}[]
\CommentTok{# leaving only "scientific name" rows}
\NormalTok{ncbi_taxon_names }\OperatorTok
\StringTok{  }\KeywordTok{filter}\NormalTok{(type }\OperatorTok{==}\StringTok{ "scientific name"}\NormalTok{) }\OperatorTok
\StringTok{  }\KeywordTok{select}\NormalTok{(name, ncbi_name)}

\CommentTok{# finding Eukaryota taxid}
\NormalTok{eukaryota_taxon_id <-}\StringTok{ }\KeywordTok{subset}\NormalTok{(ncbi_taxon_names, ncbi_name }\OperatorTok{==}\StringTok{ "Eukaryota"}\NormalTok{, }\StringTok{"name"}\NormalTok{, }\DataTypeTok{drop =} \OtherTok{TRUE}\NormalTok{)}

\CommentTok{# creating graph}
\NormalTok{g <-}\StringTok{ }\KeywordTok{graph_from_data_frame}\NormalTok{(ncbi_edgelist[,}\DecValTok{2}\OperatorTok{:}\DecValTok{1}\NormalTok{], }\DataTypeTok{directed =} \OtherTok{TRUE}\NormalTok{, }\DataTypeTok{vertices =}\NormalTok{ ncbi_taxon_names)}

\CommentTok{# easing memory}
\KeywordTok{rm}\NormalTok{(ncbi_edgelist, ncbi_merged_ids)}
\end{Highlighting}
\end{Shaded}

\hypertarget{traversing-the-graph}{%
\paragraph{\texorpdfstring{\textbf{Traversing the
graph}}{Traversing the graph}}\label{traversing-the-graph}}

\texttt{}\\
The second step is to traverse the graph from the Eukaryota root node to
STRING species nodes. This automatically drops all non-eukaryotes and
results in a species tree representing only STRING eukaryotes (476).

\begin{Shaded}
\begin{Highlighting}[]
\NormalTok{eukaryote_root <-}\StringTok{ }\KeywordTok{V}\NormalTok{(g)[eukaryota_taxon_id]}
\NormalTok{eukaryote_leaves <-}\StringTok{ }\KeywordTok{V}\NormalTok{(g)[string_species[[}\StringTok{"new_taxid"}\NormalTok{]]]}

\CommentTok{# not_found <- subset(string_species, !new_taxid %in% ncbi_taxon_names$name)}

\NormalTok{eukaryote_paths <-}\StringTok{ }\KeywordTok{shortest_paths}\NormalTok{(g, }\DataTypeTok{from =}\NormalTok{ eukaryote_root, }\DataTypeTok{to =}\NormalTok{ eukaryote_leaves, }\DataTypeTok{mode =} \StringTok{"out"}\NormalTok{)}\OperatorTok{$}\NormalTok{vpath}

\NormalTok{eukaryote_vertices <-}\StringTok{ }\NormalTok{eukaryote_paths }\OperatorTok\StringTok{ }\NormalTok{unlist }\OperatorTok\StringTok{ }\NormalTok{unique}

\NormalTok{eukaryote_tree <-}\StringTok{ }\KeywordTok{induced_subgraph}\NormalTok{(g, eukaryote_vertices, }\DataTypeTok{impl =} \StringTok{"create_from_scratch"}\NormalTok{)}
\end{Highlighting}
\end{Shaded}

\hypertarget{saving}{%
\paragraph{\texorpdfstring{\textbf{Saving}}{Saving}}\label{saving}}

\texttt{}\\
Saving \texttt{ncbi\_tree} and \texttt{string\_eukaryotes} for package
use. These data files are documented by the package. We also create a
plain text file \texttt{476\_ncbi\_eukaryotes.txt} containing the
updated names of all 476 STRING eukaryotes. This file will be queried
against the TimeTree website.

\begin{Shaded}
\begin{Highlighting}[]
\NormalTok{ncbi_tree <-}\StringTok{ }\NormalTok{treeio}\OperatorTok{::}\KeywordTok{as.phylo}\NormalTok{(eukaryote_tree)}

\CommentTok{# plot(ncbi_tree %>% ape::ladderize(), type="cladogram")}

\NormalTok{string_eukaryotes <-}\StringTok{ }\NormalTok{string_species }\OperatorTok
\StringTok{  }\KeywordTok{filter}\NormalTok{(new_taxid }\OperatorTok\StringTok{ }\NormalTok{ncbi_tree}\OperatorTok{$}\NormalTok{tip.label) }\OperatorTok
\StringTok{  }\KeywordTok{inner_join}\NormalTok{(ncbi_taxon_names, }\DataTypeTok{by =} \KeywordTok{c}\NormalTok{(}\StringTok{"new_taxid"}\NormalTok{ =}\StringTok{ "name"}\NormalTok{))}

\KeywordTok{write}\NormalTok{(string_eukaryotes[[}\StringTok{"ncbi_name"}\NormalTok{]],}\StringTok{"476_ncbi_eukaryotes.txt"}\NormalTok{)}

\CommentTok{# usethis::use_data(ncbi_tree, overwrite = TRUE)}
\KeywordTok{write.tree}\NormalTok{(ncbi_tree, }\StringTok{"tree_ncbi.nwk"}\NormalTok{)}
\NormalTok{usethis}\OperatorTok{::}\KeywordTok{use_data}\NormalTok{(string_eukaryotes, }\DataTypeTok{overwrite =} \OtherTok{TRUE}\NormalTok{)}
\end{Highlighting}
\end{Shaded}

\begin{verbatim}
## <U+2714> Setting active project to 'C:/R/neuro'
## <U+2714> Saving 'string_eukaryotes' to 'data/string_eukaryotes.rda'
\end{verbatim}
