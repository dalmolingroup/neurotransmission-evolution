\hypertarget{expression-neuroexclusivity}{%
\subsubsection{Expression
neuroexclusivity}\label{expression-neuroexclusivity}}

Multiple wide .tsv files are preprocessed into a single long data frame.
We also create a template file for manually classifying tissues into
nervous or non-nervous categories.

\textbf{Resources}\\
We start by searching Gene Expression Atlas for experiments that have
human baseline expression data at the tissue level. For each experiment,
TPM expression data is downloaded to the \texttt{data-raw/download/gxa/}
directory. The following 8 experiments could be found (hyperlinked):

\begin{itemize}
\tightlist
\item
  \href{https://www.ebi.ac.uk/gxa/experiments/E-MTAB-513}{E-MTAB-513}
\item
  \href{https://www.ebi.ac.uk/gxa/experiments/E-MTAB-2836}{E-MTAB-2836}
\item
  \href{https://www.ebi.ac.uk/gxa/experiments/E-MTAB-3358}{E-MTAB-3358}
\item
  \href{https://www.ebi.ac.uk/gxa/experiments/E-MTAB-3708}{E-MTAB-3708}
\item
  \href{https://www.ebi.ac.uk/gxa/experiments/E-MTAB-3716}{E-MTAB-3716}
\item
  \href{https://www.ebi.ac.uk/gxa/experiments/E-MTAB-4344}{E-MTAB-4344}
\item
  \href{https://www.ebi.ac.uk/gxa/experiments/E-MTAB-4840}{E-MTAB-4840}
\item
  \href{https://www.ebi.ac.uk/gxa/experiments/E-MTAB-5214}{E-MTAB-5214}
\end{itemize}

\textbf{Reshaping}\\
Loading and pivoting all data to a long format.

\begin{Shaded}
\begin{Highlighting}[]
\CommentTok{# Loading}
\NormalTok{gene_expression <-}\StringTok{ }\KeywordTok{sapply}\NormalTok{(}
  \KeywordTok{list.files}\NormalTok{(}\StringTok{"download/gxa/"}\NormalTok{, }\DataTypeTok{full.names =}\NormalTok{ T),}
\NormalTok{  read_tsv,}
  \DataTypeTok{comment =} \StringTok{"#"}\NormalTok{,}
  \DataTypeTok{simplify =} \OtherTok{FALSE}\NormalTok{,}
  \DataTypeTok{USE.NAMES =} \OtherTok{TRUE}
\NormalTok{)}

\CommentTok{# Pivoting}
\NormalTok{gene_expression }\OperatorTok
\StringTok{  }\KeywordTok{map_dfr}\NormalTok{(pivot_longer, }\DataTypeTok{cols =} \OperatorTok{-}\NormalTok{(}\DecValTok{1}\OperatorTok{:}\DecValTok{2}\NormalTok{), }\DataTypeTok{names_to =} \StringTok{"tissue"}\NormalTok{, }\DataTypeTok{values_to =} \StringTok{"tpm"}\NormalTok{) }\OperatorTok
\StringTok{  }\NormalTok{na.omit }\OperatorTok
\StringTok{  }\KeywordTok{select}\NormalTok{(}\DataTypeTok{ensembl_id =} \StringTok{`}\DataTypeTok{Gene ID}\StringTok{`}\NormalTok{, tissue, tpm)}
\end{Highlighting}
\end{Shaded}

\textbf{Cleaning}\\
A lot of tissue annotation can be collapsed into single levels
(e.g.~``brain'' and ``brain fragment'' can be considered the same
tissue). The cleaning is performed and expression data is exported for
analysis.

\begin{Shaded}
\begin{Highlighting}[]
\CommentTok{# E-MTAB-4840 has comma separated developmental stage info (removing everything before ", ")}
\NormalTok{gene_expression }\OperatorTok\StringTok{ }\KeywordTok{mutate}\NormalTok{(}\DataTypeTok{tissue =} \KeywordTok{str_remove}\NormalTok{(tissue, }\StringTok{"^.+, "}\NormalTok{))}

\NormalTok{tissue_names_fix <-}\StringTok{ }\KeywordTok{c}\NormalTok{(}
  \StringTok{"brain fragment"}\NormalTok{                     =}\StringTok{ "brain"}\NormalTok{,}
  \StringTok{"forebrain fragment"}\NormalTok{                 =}\StringTok{ "forebrain"}\NormalTok{,}
  \StringTok{"forebrain and midbrain"}\NormalTok{             =}\StringTok{ "forebrain"}\NormalTok{,}
  \StringTok{"hindbrain fragment"}\NormalTok{                 =}\StringTok{ "hindbrain"}\NormalTok{,}
  \StringTok{"hindbrain without cerebellum"}\NormalTok{       =}\StringTok{ "hindbrain"}\NormalTok{,}
  \StringTok{"hippocampus proper"}\NormalTok{                 =}\StringTok{ "hippocampus"}\NormalTok{,}
  \StringTok{"hippocampal formation"}\NormalTok{              =}\StringTok{ "hippocampus"}\NormalTok{,}
  \StringTok{"diencephalon and midbrain"}\NormalTok{          =}\StringTok{ "diencephalon"}\NormalTok{,}
  \StringTok{"visceral (omentum) adipose tissue"}\NormalTok{  =}\StringTok{ "adipose tissue"}\NormalTok{,}
  \StringTok{"subcutaneous adipose tissue"}\NormalTok{        =}\StringTok{ "adipose tissue"}\NormalTok{,}
  \StringTok{"spinal cord (cervical c-1)"}\NormalTok{         =}\StringTok{ "spinal cord"}\NormalTok{,}
  \StringTok{"C1 segment of cervical spinal cord"}\NormalTok{ =}\StringTok{ "spinal cord"}
\NormalTok{)}

\NormalTok{gene_expression }\OperatorTok\StringTok{ }\KeywordTok{mutate}\NormalTok{(}\DataTypeTok{tissue =} \KeywordTok{recode}\NormalTok{(tissue, }\OperatorTok{!!!}\NormalTok{tissue_names_fix))}
           
\CommentTok{# Subseting for genes of interest}
\NormalTok{gene_expression }\OperatorTok\StringTok{ }\KeywordTok{filter}\NormalTok{(ensembl_id }\OperatorTok\StringTok{ }\NormalTok{gene_ids[[}\StringTok{"ensembl_id"}\NormalTok{]])}

\CommentTok{# Exporting for package use}
\NormalTok{usethis}\OperatorTok{::}\KeywordTok{use_data}\NormalTok{(gene_expression, }\DataTypeTok{overwrite =} \OtherTok{TRUE}\NormalTok{)}
\end{Highlighting}
\end{Shaded}

\textbf{Tissue classification}\\
For subsequent analyses, we need to distinguish if a tissue is part of
the nervous system or not. This is done by hand. The first step is to
write a temp file to
\texttt{data-raw/temp/temp\_tissue\_classification.tsv} with all tissue
names. This serves as a base for the completed
\texttt{data/neuroexclusivity\_classification\_tissue} file.

\begin{Shaded}
\begin{Highlighting}[]
\NormalTok{gene_expression }\OperatorTok
\StringTok{  }\KeywordTok{select}\NormalTok{(tissue) }\OperatorTok
\StringTok{  }\NormalTok{unique }\OperatorTok
\StringTok{  }\NormalTok{arrange }\OperatorTok
\StringTok{  }\KeywordTok{mutate}\NormalTok{(}\DataTypeTok{is_nervous =} \OtherTok{NA}\NormalTok{) }\OperatorTok
\StringTok{  }\KeywordTok{write_tsv}\NormalTok{(}\StringTok{"temp/temp_tissue_classification.tsv"}\NormalTok{)}
\end{Highlighting}
\end{Shaded}

\hypertarget{pathway-neuroexclusivity}{%
\subsubsection{Pathway
neuroexclusivity}\label{pathway-neuroexclusivity}}

In this section we create a template file for classifying pathways into
nervous or non-nervous.

\textbf{Resources} For \texttt{link\_pathway\_entrez} see
\hyperref[tab:link_pathway_entrez]{Table 5}.

\begin{table}[H]

\caption{\label{tab:pathway_names}KEGG pathway names.}
\resizebox{\linewidth}{!}{
\begin{tabular}[t]{rlllll}
\toprule
\multicolumn{6}{c}{\bgroup\fontsize{12}{14}\selectfont \cellcolor[HTML]{EEEEEE}{\ttfamily{\textbf{pathway\_names}}}\egroup{}} \\
\cmidrule(l{3pt}r{3pt}){1-6}
\# & Col. name & Col. type & Used? & Example & Description\\
\midrule
\rowcolor{gray!6}  1 & pathway\_id & character & yes & path:hsa04726 & KEGG pathway ID\\
2 & pathway\_name & character & yes & Serotonergic synapse - Homo sapiens (human) & pathway name\\
\bottomrule
\multicolumn{6}{l}{\textbf{Location: } data-raw/download/pathway\_names.tsv}\\
\multicolumn{6}{l}{\textbf{Source: } http://rest.kegg.jp/list/pathway/hsa}\\
\end{tabular}}
\end{table}

\textbf{Pathway classification}\\
Just like tissues, we need to distinguish if a pathway is related to the
nervous system or not. This is done by hand. The first step is to write
a temp file to \texttt{data-raw/temp/temp\_pathway\_classification.tsv}
with all pathway names. This serves as a base for the completed
\texttt{data/neuroexclusivity\_classification\_pathway.tsv} file.

\begin{Shaded}
\begin{Highlighting}[]
\CommentTok{# Removing species prefix ("hsa:")}
\NormalTok{link_pathway_entrez[[}\StringTok{"entrez_id"}\NormalTok{]] }\OperatorTok\StringTok{ }\KeywordTok{str_split_n}\NormalTok{(}\StringTok{"}\CharTok{\textbackslash{}\textbackslash{}}\StringTok{:"}\NormalTok{, }\DecValTok{2}\NormalTok{)}

\NormalTok{selected_genes_pathways <-}\StringTok{ }\NormalTok{link_pathway_entrez }\OperatorTok\StringTok{ }\KeywordTok{filter}\NormalTok{(entrez_id }\OperatorTok\StringTok{ }\NormalTok{gene_ids[[}\StringTok{"entrez_id"}\NormalTok{]])}

\NormalTok{unique_pathway_ids <-}\StringTok{ }\NormalTok{selected_genes_pathways }\OperatorTok\StringTok{ }\KeywordTok{pull}\NormalTok{(pathway_id) }\OperatorTok\StringTok{ }\NormalTok{unique}

\NormalTok{pathway_names }\OperatorTok\StringTok{ }\KeywordTok{filter}\NormalTok{(pathway_id }\OperatorTok\StringTok{ }\NormalTok{unique_pathway_ids) }\OperatorTok
\StringTok{  }\KeywordTok{mutate}\NormalTok{(}\DataTypeTok{is_nervous =} \OtherTok{NA}\NormalTok{) }\OperatorTok
\StringTok{  }\KeywordTok{write_tsv}\NormalTok{(}\StringTok{"temp/temp_pathway_classification.tsv"}\NormalTok{)}
\end{Highlighting}
\end{Shaded}

