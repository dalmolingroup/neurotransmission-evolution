\begin{table}[H]

\caption{\label{tab:cogs}Orthologous groups (COGs, NOGs, KOGs) and their proteins.}
\resizebox{\linewidth}{!}{
\begin{tabular}[t]{rllll>{\raggedright\arraybackslash}p{18em}}
\toprule
\multicolumn{6}{c}{\bgroup\fontsize{12}{14}\selectfont \cellcolor[HTML]{EEEEEE}{\ttfamily{\textbf{cogs}}}\egroup{}} \\
\cmidrule(l{3pt}r{3pt}){1-6}
\# & Col. name & Col. type & Used? & Example & Description\\
\midrule
\rowcolor{gray!6}  1 & taxid.string\_id & character & yes & 9606.ENSP00000269305 & STRING protein ID\\
2 & start\_position & numeric & no & 1 & residue where orthogroup mapping starts\\
\rowcolor{gray!6}  3 & end\_position & numeric & no & 393 & residue where orthogroup mapping ends\\
4 & cog\_id & character & yes & NOG08732 & orthologous group ID\\
\rowcolor{gray!6}  5 & protein\_annotation & character & no & Cellular tumor antigen p53; [...] & protein description\\
\bottomrule
\multicolumn{6}{l}{\textbf{Location: } data-raw/download/COG.mappings.v11.0.txt.gz}\\
\multicolumn{6}{l}{\textbf{Source: } https://stringdb-static.org/download/COG.mappings.v11.0.txt.gz}\\
\end{tabular}}
\end{table}

\begin{Shaded}
\begin{Highlighting}[]
\CommentTok{# Spliting first column into taxid and string_id}
\NormalTok{separated_ids <-}\StringTok{ }\NormalTok{cogs }\OperatorTok\StringTok{ }\KeywordTok{stri_split_fixed}\NormalTok{(taxid.string_id, }\DataTypeTok{pattern =} \StringTok{"."}\NormalTok{, }\DataTypeTok{n =} \DecValTok{2}\NormalTok{, }\DataTypeTok{simplify =}\NormalTok{ T)}

\NormalTok{cogs[[}\StringTok{"taxid"}\NormalTok{]]     <-}\StringTok{ }\NormalTok{separated_ids[, }\DecValTok{1}\NormalTok{]}
\NormalTok{cogs[[}\StringTok{"string_id"}\NormalTok{]] <-}\StringTok{ }\NormalTok{separated_ids[, }\DecValTok{2}\NormalTok{]}

\CommentTok{# Freeing up some memory}
\KeywordTok{rm}\NormalTok{(separated_ids)}
\KeywordTok{gc}\NormalTok{()}

\CommentTok{# keeping only eukaryotes}
\NormalTok{cogs }\OperatorTok\StringTok{ }\KeywordTok{select}\NormalTok{(}\OperatorTok{-}\NormalTok{taxid.string_id) }\OperatorTok\StringTok{ }\KeywordTok{filter}\NormalTok{(taxid }\OperatorTok\StringTok{ }\NormalTok{string_eukaryotes[[}\StringTok{"taxid"}\NormalTok{]])}
\KeywordTok{gc}\NormalTok{()}

\CommentTok{# Subsetting cogs of interest}
\NormalTok{gene_cogs <-}\StringTok{ }\NormalTok{cogs }\OperatorTok
\StringTok{    }\KeywordTok{filter}\NormalTok{(string_id }\OperatorTok\StringTok{ }\NormalTok{gene_ids[[}\StringTok{"string_id"}\NormalTok{]]) }\OperatorTok
\StringTok{    }\KeywordTok{select}\NormalTok{(}\OperatorTok{-}\NormalTok{taxid) }\OperatorTok
\StringTok{    }\KeywordTok{group_by}\NormalTok{(string_id) }\OperatorTok
\StringTok{    }\KeywordTok{summarise}\NormalTok{(}\DataTypeTok{n =} \KeywordTok{n}\NormalTok{(), }\DataTypeTok{cog_id =} \KeywordTok{paste}\NormalTok{(cog_id, }\DataTypeTok{collapse =} \StringTok{" / "}\NormalTok{))}
\end{Highlighting}
\end{Shaded}

\hypertarget{proteins-with-multiple-cogs}{%
\subsubsection{Proteins with multiple
COGs}\label{proteins-with-multiple-cogs}}

Some proteins are occasionally assigned to more than one orthologous
group (OG) by the COG algorithm. Despite infrequent, a decision needs to
be made to deal with these exceptions. Since we aim to identify the most
ancient archetype that has been vertically inherited to humans, it is
reasonable to resolve these proteins to their most ancient OG (in the
case of a human protein assigned to more than one OG). However, this
strategy can sometimes be inappropriate for multidomain proteins. For
instance, the SHANK proteins are assigned to two OG, of which the most
ancient one is simply associated with ankyrin repeats. A similar case
involves the PLA2G4 family and C2 domains, where the six members are
assigned to more than one OG. We decided to resolve SHANK and PLA2G4
proteins to their most recent OGs.

\begin{Shaded}
\begin{Highlighting}[]
\NormalTok{gene_cogs }\OperatorTok\StringTok{ }\KeywordTok{filter}\NormalTok{(n }\OperatorTok{>}\StringTok{ }\DecValTok{1}\NormalTok{)}
\end{Highlighting}
\end{Shaded}

\begin{tabular}{lrl}
\toprule
string\_id & n & cog\_id\\
\midrule
\rowcolor{gray!6}  ENSP00000290472 & 2 & KOG1028 / KOG1325\\
ENSP00000293441 & 2 & COG0666 / KOG4375\\
\rowcolor{gray!6}  ENSP00000356436 & 2 & COG5038 / KOG1325\\
ENSP00000371886 & 3 & COG1226 / KOG1028 / KOG1325\\
\rowcolor{gray!6}  ENSP00000380442 & 2 & KOG1028 / KOG1325\\
\addlinespace
ENSP00000382434 & 2 & KOG1028 / KOG1325\\
\rowcolor{gray!6}  ENSP00000396045 & 2 & KOG1028 / KOG1325\\
ENSP00000469689 & 2 & COG0666 / KOG4375\\
\bottomrule
\end{tabular}

\begin{Shaded}
\begin{Highlighting}[]
\NormalTok{gene_cogs_resolved <-}\StringTok{ }\KeywordTok{tribble}\NormalTok{(}
\CommentTok{##############################}
       \OperatorTok{~}\NormalTok{string_id,   }\OperatorTok{~}\NormalTok{cog_id,}\CommentTok{#}
\StringTok{"ENSP00000356436"}\NormalTok{, }\StringTok{"KOG1325"}\NormalTok{,}\CommentTok{# PLA2G4A}
\StringTok{"ENSP00000396045"}\NormalTok{, }\StringTok{"KOG1325"}\NormalTok{,}\CommentTok{# PLA2G4B}
\StringTok{"ENSP00000290472"}\NormalTok{, }\StringTok{"KOG1325"}\NormalTok{,}\CommentTok{# PLA2G4D}
\StringTok{"ENSP00000382434"}\NormalTok{, }\StringTok{"KOG1325"}\NormalTok{,}\CommentTok{# PLA2G4E}
\StringTok{"ENSP00000380442"}\NormalTok{, }\StringTok{"KOG1325"}\NormalTok{,}\CommentTok{# PLA2G4F}
\StringTok{"ENSP00000371886"}\NormalTok{, }\StringTok{"KOG1325"}\NormalTok{,}\CommentTok{# JMJD7-PLA2G4B}
\StringTok{"ENSP00000293441"}\NormalTok{, }\StringTok{"KOG4375"}\NormalTok{,}\CommentTok{# SHANK1}
\StringTok{"ENSP00000469689"}\NormalTok{, }\StringTok{"KOG4375"}\NormalTok{,}\CommentTok{# SHANK2}
\NormalTok{)}\CommentTok{#############################}

\NormalTok{gene_cogs }\OperatorTok
\StringTok{  }\CommentTok{# Removing unresolved cases}
\StringTok{  }\KeywordTok{filter}\NormalTok{(n }\OperatorTok{==}\StringTok{ }\DecValTok{1}\NormalTok{) }\OperatorTok
\StringTok{  }\KeywordTok{select}\NormalTok{(}\OperatorTok{-}\NormalTok{n) }\OperatorTok
\StringTok{  }\CommentTok{# Adding the manual assignments}
\StringTok{  }\KeywordTok{bind_rows}\NormalTok{(gene_cogs_resolved)}
\end{Highlighting}
\end{Shaded}

\begin{Shaded}
\begin{Highlighting}[]
\CommentTok{# Exporting for package use}
\NormalTok{usethis}\OperatorTok{::}\KeywordTok{use_data}\NormalTok{(cogs, }\DataTypeTok{overwrite =} \OtherTok{TRUE}\NormalTok{)}
\NormalTok{usethis}\OperatorTok{::}\KeywordTok{use_data}\NormalTok{(gene_cogs, }\DataTypeTok{overwrite =} \OtherTok{TRUE}\NormalTok{)}
\end{Highlighting}
\end{Shaded}

